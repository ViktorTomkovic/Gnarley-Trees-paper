% !Mode:: "TeX::UTF-8"
\documentclass[slovensky]{svk}
\usepackage[round]{natbib}
\usepackage{subfig}
\usepackage{array}

\begin{document}
  \catcode`\"=\active \def "{\begingroup\clqq\def "{\endgroup\crqq}}

\title{Gnarley Trees}

\author{
  Katka Kotrlová % \email{katkin email, ktorý chce zverejniť}
  \and 
  Pavol Lukča % \email{palyho mail}
  \and
  Viktor Tomkovič % \email{viktor.tomkovic@gmail.com}
  \and 
  Tatiana Tóthová % \email{táničkin mail}
}

\supervisor{
  Jakub Kováč %\email{kuko@ksp.sk}
  \email{algvis@googlegroups.com}
}

%% nasleduje kratka verzia nazvu clanku a 
%% zoznam autorov (bez krstnych mien)
%% tieto informacie sa zobrazuju v hlavicke
\titlerunning{Gnarley Trees}
\authorrunning{Kotrlová, Lukča, Tomkovič, Tóthová}

\institute{
Katedra informatiky,
FMFI UK,
Mlynská Dolina,
842~48~Bratislava}

\maketitle

\begin{abstract}
V tomto článku prezentujeme našu prácu na projekte Gnarley Trees, ktorý 
začal Jakub Kováč ako svoju bakalársku prácu. Gnarley Trees je 
projekt, ktorý má dve časti. Prvá časť sa zaoberá kompiláciou dátových 
štruktúr, ktoré majú stromovú štruktúru, ich popisom a popisom ich 
hlavných výhod a nevýhod oproti iným dátovým štruktúram. Druhá časť sa 
zaoberá ich vizualizáciou a vizualizáciou vybraných algoritmoch na 
týchto štruktúrach.

\noindent
\textbf{Dostupnosť:} Softvér je voľne dostupný na stránke
                     \url{http://people.ksp.sk/~kuko/gnarley-trees}.
\keywords{Gnarley Trees, vizualizácia, algoritmy a dátové štruktúry}
\end{abstract}

\section{Úvod}
Ako ľudia so záujmom o dátové štruktúry sme sa rozhodli pomôcť vybudovať 
dobrý softvér na vizualizáciu algoritmov a dátových štruktúr a obohatiť 
kompiláciu Jakuba Kováča \citep{kuko} o ďalšie dátové štruktúry. 
Vizualizujeme rôznorodé dátové štruktúry. Z binárnych vyvažovaných stromov 
to sú \emph{finger tree} a \emph{reversal tree}, z háld to sú \emph{d-nárna 
halda}, \emph{ľavicová halda}, \emph{skew halda} a \emph{párovacia halda}. 
Taktiež vizualizujeme aj \emph{problém disjuktných množín (union-find 
problém)} a \emph{písmenkový strom (trie)}. 

Okrem vizualizácie prerábame softvér, doplnili sme ho o históriu krokov 
a operácií, jednoduchšie ovládanie a veľa iných vecí, zlepšujúcich 
celkový dojem. Softvér je celý v slovenčine a angličtine a je 
implementovaný v jazyku \texttt{Java}.

\subsection{Vizualizácia}
Dátové štruktúry a algoritmy tvoria základnú, prvotnú časť výučby 
informatiky. Vizualizácia algoritmov a dátových štruktúr je grafické 
znázornenie, ktoré abstrahuje spôsob ako algoritmus a dátové štruktúry 
pracujú od ich vnútornej reprezentácie a umiestnení v pamäti. Je teda 
vyhľadávaná a všeobecne rozšírená pomôcka pri výučbe. Výsledky výskumov 
ohľadne jej efektívnosti sa líšia, od stavu \uv{nezaznamenali sme výrazné 
zlepšenie} po \uv{je viditeľné zlepšenie}. \citep{shaffer}

Rozmach vizualizačných algoritmov priniesla najmä Java a jej fungovanie 
bez viazanosti na konkrétny operačný systém. Kvalita vizualizácií sa líši 
a keďže ide o ľahko naprogramovateľné programy, je ich veľa a sú pomerne 
nekvalitné. V takomto množstve je ťažké nájsť kvalitné vizualizácie. 
Zbieraním a analyzovaním kvality sa venuje skupina AlgoViz, ktorá už 
veľa rokov funguje na portále \url{http://algoviz.org/}.

Zaujímavé je pozorovanie, že určovanie si vlastného tempa pri vizualizácií 
je veľká pomôcka. Naopak, ukazovanie pseudokódu alebo nemožnosť určenia si
vlastného tempa (napríklad animácia bez možnosti pozastavenia), takmer 
žiadne zlepšenie neprináša. \citep{shaffer,saraiya}

\subsection*{Motivácia}
Z vyššie uvedeného je jasné, že našou snahou je vytvoriť kvalitnú kompiláciu 
a softvér, ktorý bude nezávislý od operačného systému, bude vyhovovať ako 
pomôcka pri výučbe ako aj pri samoštúdiu a bude voľne prístupný a náležite 
propagovaný. Toto sú hlavné body, ktoré nespĺňa žiaden slovenský a 
len veľmi málo svetových vizualizačných softvérov. Našou hlavnou snahou 
je teda ponúknuť plnohodnotné prostredie pri učení.
%co robime? na co je dobra vizualizacia? (mozete porovnat svet s a bez vizualizacie :))
%co konkretne vizualizujeme? preco to robime? (uz daco take je? [da sa odcitovat shaffer])
%
%Friker, makaj! Zvyšok, pomôž!
\section{Rozšírenie predošlej práce}

\begin{figure*}
\includegraphics[width=2\columnwidth]{obrazky/historia.png}
\caption{\emph{Softvér Gnarley Trees.} Vpravo je história krokov. V histórií 
sa dá navigovať buď pomocou tlačidiel \uv{Späť}/\uv{Ďalej}, alebo kliknutím na 
konkrétny krok v histórií.}
\label{img:historia} 
\end{figure*}

Projekt Gnarley Trees sme rozšírili nielen o vizualizácie ďalších dátových
štruktúr, ale pribudli aj softvérové (vizualizačné) vylepšenia:
kompaktnejšie vykresľovanie stromov, história krokov s možnosťou návratu,
či približovanie a vzďaľovanie.

\subsection{Tesnejšie vykresľovanie stromov}

V pôvodnej verzii programu sa stromy vykresľovali tak, že vertikálna súradnica 
predstavovala hĺbku v strome a horizontálna poradie vrcholu v \emph{
inorderovom prechode} stromu. Tento jednoduchý spôsob však nešetrí priestor
a pri štruktúrach ako union-find či písmenkový strom by výsledné stromy 
vyzerali škaredo. Preto sme sa rozhodli pre stromy implementovať 
alternatívny spôsob rozloženia, ktorý pre binárne stromy navrhli \citet{reingold}
a na všeobecné stromy rozšíril \citet{walker}. Tieto rozloženia vykresľujú
vrcholy stromov čo najtesnejšie, pričom dodržujú tieto estetické pravidlá: 
\begin{itemize} 
\item vrcholy v rovnakej hĺbke sú vykreslené na jednej priamke a priamky 
určujúce jednotlivé úrovne sú rovnobežné; 
\item poradie synov je zachované; 
\item otec leží v strede nad najľavejším a najpravejším synom; 
\item izomorfné podstromy sa vykreslia identicky až na presunutie;
\item ak vo všetkých vrcholoch obrátime poradie všetkých synov, výsledný strom 
sa vykreslí zrkadlovo.
\end{itemize}
Reingoldov-Tilfordov, aj zovšeobecnený Walkerov algoritmus pracuje v lineárnom
čase.

\section{História}
Každá vizualizovaná operácia (insert/delete/...) na dátovej štruktúre pozostáva z niekoľkých krokov. Jednou z noviniek v projekte je možnosť vrátiť sa pri prehliadani operácií o niekoľko krokov späť (história krokov), resp. vrátiť späť celé operácie (história operácií).\\
Niekedy sa stáva, že nedočkavý užívateľ rýchlo prekliká cez celú vizualizáciu operácie a pritom si nestihne uvedomiť, aké zmeny sa vykonali na danej dátovej štruktúre. Inokedy je operácia taká rozsiahla, že niektoré dôležité zmeny si nevšimne. Vtedy by bolo užitočné pozrieť si vizualizáciu ešte raz (alebo niekoľkokrát). Tento problém rieši história krokov. Užívateľ má možnosť vrátiť sa späť o jeden krok (tlačidlo "Späť"/"Previous") alebo preskočiť na ľubovoľný krok po kliknutí na zodpovedajúci komentár. \\
História krokov a operácií je atomická. Krok/operácia sa vykoná/vráti celý(-á) alebo vôbec, pričom stav dátovej štruktúry korešponduje s pozíciou v histórii. To umožňuje po vrátení celej operácie vykonať inú operáciu. Táto vlastnosť je užitočná najmä v prípade vykonania operácie (príp. zmazania celej dátovej štruktúry) omylom.


\subsection{Ďalšie rozšírenia}
Patrí k nim
možnosť priblíženia/vzdialenia a presunu vykreslenej dátovej štruktúry v rámci
vizualizačnej plochy. Užívateľ túto funkcionalitu využije najmä pri dátových
štruktúrach s veľkým počtom prvkov, kedy je obmedzený veľkosťou plochy. Ďalším
rozšírením je výpis celej postupnosti komentárov vizualizovanej operácie,
ktorý prináša spolu s históriou krokov značné zjednodušenie
výučby. Užívateľ si môže konkrétnu vizualizáciu pozrieť toľko krát, koľko
potrebuje na jej správne pochopenie. Navyše vidí, aké kroky budú
nasledovať/predchádzať a podľa toho si môže určiť vlastné tempo prezerania
vizualizácie. Ak si myslí, že daným krokom už porozumel, môže ich preskočiť.


\def\reverse{$\mathop{\mathit{reverse}}(i, j)$}
\def\Bp{$\hbox{\rm B}^+$}

\section{Vyvážené stromy}

Pôvodná verzia \citep{kuko} obsahovala vizualizáciu viacerých vyvážených stromov.
K nim sme pridali \Bp-stromy, stromy s prstom a stromy s reverzami.

\subsection{B$^+$-strom}

\paragraph{Popis.}
\emph{\Bp-strom} je variácia B-stromu, v ktorom sú všetky kľúče uložené v listoch
a listy sú pospájané do spájaného zoznamu. \Bp-strom rádu $B$ je strom, v ktorom
má každý vnútorný vrchol najmenej $\lfloor B/2 \rfloor$, ale najviac $B$ synov.
Vďaka tomu je dobre vyvážený a jeho operácie sú vykonávané v logaritmickom čase.
\Bp-strom je \emph{asociatívne pole (slovník)}, čiže poskytuje tieto tri operácie:
\begin{itemize}
\item $\ins(x)$ -- pridá do stromu $x$;
\item $\find(x)$ -- zistí, či sa v strome nachádza $x$;
\item $\delete(x)$ -- odstráni zo stromu $x$.
\end{itemize}

Operácia $\find$ začne v koreni, nájde v ňom prvý kľúč väčší od hľadaného.
Nech je $i$-ty v poradí, potom hľadanie pokračuje v $i$-tom synovi tohto vrcholu.
Je zrejmé, že ak sa väčší kľúč nenájde, presunieme sa do posledného, $B$-tého syna.
V liste sa už len skontroluje, či sa v ňom hľadaný kľúč nachádza.

Definujeme dve operácie: {\sc Copy-Up} a {\sc Push-Up}, ktoré používa operácia $\ins$.
Ak má vrchol viac prvkov, ako je maximálny limit, treba ho zmenšiť. Rozdelí sa na dve časti.
Ak vrchol nie je listom, použije sa {\sc Push-Up}, najmenší kľúč pravej časti sa vyberie
a stane sa otcom vytvorených dvoch častí. Pokiaľ to list je, kľúč v ňom musí zostať,
preto sa iba skopíruje. Táto operácia sa nazýva {\sc Copy-Up}.

Operácia $\ins$ najprv pomocou operácie $\find$ zistí, či štruktúra daný kľúč obsahuje.
Ak nie, je zrejmé, že patrí práve do vrchola, kde $\find$ skončil. Ak má vrchol po vložení
viac kľúčov ako maximálny limit, je treba ho zmenšiť. Pokiaľ otcovský vrchol má syna,
do ktorého je možné kľúč presunúť a zároveň syn susedí s veľkým vrcholom, postup je nasledovný.
Nech je vrchol vľavo menší a terajší vrchol je $i$-ty syn v poradí. Potom algoritmus z neho
vyberie najmenší kľúč, presunie ho na miesto $(i-1)$-ho kľúča v otcovskom vrchole. Vymenený
kľúč následne vloží do $(i-1)$-ho syna. Ak sú susedné vrcholy príliš veľké na presun kľúča,
použije sa operácia {\sc Copy-Up}. Nový vrchol s jedným kľúčom, ktorý vznikol, vložíme do
otcovského vrcholu. Ak otcovský vrchol presiahol najväčšiu možnú veľkosť, znova sa aplikuje
popísaný algoritmus s jedným rozdielom -- namiesto {\sc Copy-Up} sa použije {\sc Push-Up}.

Operácia $\delete$ najprv pomocou $\find$ nájde kľúč, potom ho z vrcholu odstráni. Tento vrchol
môže mať po odstránení menší počet kľúčov ako minimálny limit. Vtedy, ak sa dá, sa prenesie
jeden kľúč zo súrodenca. Ak sa nedá, vrchol sa s ním zlúči. Zároveň sa k nim pridá aj kľúč
z otcovského vrcholu, ktorý ich rozdeľoval. Pokiaľ to spôsobilo, že otcovský vrchol má menej
kľúčov, ako je povolené, znova sa aplikuje predošlý algoritmus. Keďže na koreň sa nevzťahuje
minimálny limit, po skončení bude strom zaručene v konzistentnom tvare.

%\paragraph{Časová zložitosť}
%\todo{toto je pocet pristupov na disk! pocet krokov je $O(B\log_B n)$; ale pocet pristupov
%na disk nas zaujima, lebo disk je pomaly}
Vkladanie, vymazávanie a hľadanie má časovú zložitosť $O(\log_B n)$.

\paragraph{Použitie.}
Hlavné využitie \Bp-stromov je v databázových systémoch. Ak zvolíme vhodný rád, vieme jednotlivé
vrcholy dobre napasovať na stránky a tým regulovať ako počet prístupov k pamäti, tak jej zaplnenie
\citep{sahni}. Agregačné funkcie, ako napríklad súčet, minimum, priemer, vieme pre daný interval
spočítať v čase $O(\log_B(n))$. Vypísať všetky prvky z daného intervalu dokážeme pomocou
$O(\log_B(n) + t/B)$ prístupov na disk.
% \Bp-strom podporuje efektívne vyhľadanie prvkov poľa, ktoré patria do daného intervalu.
% Algoritmus nájde jeden krajný bod a vďaka spájanému zoznamu, vytvorenému z listov, ostatné prvky
% postupne prečíta. Zložitosť je $O(\log_B(n) + t/B)$, kde $t$ je počet výsledných kľúčov z hľadaného intervalu.

Ďaľšia výhoda \Bp-stromov oproti B-stromom sa prejaví, ak máme utriedený zoznam dát a chceme z neho vytvoriť \Bp-strom:
\Bp-strom môžeme vystavať odspodu. Takýto postup nám zaručí vyžaduje $O((n/B)\cdot\log_B n)$ prístupov na disk,
čo je $B$-krát rýchlejšie ako spraviť $n$ volaní $\ins$.




\subsection{Strom s prstom}
\emph{Strom s prstom} (z anglického Finger tree) je vyhľadávací strom s hranami na vrstvách a štruktúrou 
nazývanou prst. Prst je smerník na konkrétny vrchol a umožňuje (aj vďaka väzbám navyše) efektívnejší 
prístup k vrcholom v jeho blízkom okolí.

\paragraph{Popis.}
Strom s prstom je upravený 2-3-4$^+$ strom (\Bp-strom rádu 4, t.j.\ vnútorné vrcholy
majú stupeň 2, 3 alebo 4). Kľúče sú uložené v listoch a vnútorné vrcholy
obsahujú ich kópie, aby viedli vyhľadávanie. Pre podporu vyhľadávania
s prstom spojíme všetky vrcholy na rovnakej úrovni (v rovnakej vzdialenosti od koreňa)
do obojsmerného spájaného zoznamu.
Ak sú nejaké dva vrcholy spojené takouto hranou, budeme hovoriť, že sú susedia.

Prst, ako už bolo spomenuté, ukazuje na nejaký vrchol. Môže sa pohybovať po všetkých hranách
a pomocou neho sa vykonávajú všetky operácie. Keďže sú všetky kľúče uložené v listoch, prst
na tejto vrstve začína, aj končí.

Vzhľadom na usporiadanie budem predpokladať, že menšie prvky sa nachádzajú vždy vľavo a každý
prekopírovaný kľúč má svoj originál vždy vľavo v listovom vrchole.

% Strom s prstom je \emph{asociatívne pole (slovník)}, čiže
% poskytuje tieto tri operácie:
% \begin{itemize}
% \item $\ins(\k)$ -- pridá do stromu $\k$;
% \item $\find(\k)$ -- zistí, či sa v strome nachádza $\k$;
% \item $\mathop{\mathbf{delete}}(\k)$ -- odstráni zo stromu $\k$.
% \end{itemize}

Operácia $\find$ začne na mieste, kam ukazuje prst. Skontroluje, či by kľúč mal patriť do daného
vrcholu. Ak nie, pozrie sa, či nepatrí do niektorého zo susedov. Ak áno, prst sa tam presunie
a vyhľadávanie sa skončilo. Inak smerník prejde o vrstvu vyššie, na otcovský vrchol. Ak hľadaný
kľúč patrí do jeho podstromu (t.j.\ je väčší ako jeho najmenší kľúč a menší ako ten najväčší),
zíde po hranách do listu, kde by sa daný kľúč mal nachádzať. Keď do podstromu nepatrí, skontroluje,
či nepatrí do podstromu susedov. Ak áno, prejde po vrstevnej hrane na suseda a následne zíde až
do listu, kde by mal kľúč byť. Pokiaľ prst nenarazil na správny podstrom, znova sa presunie smerom
nahor po otcovskej hrane. Hľadanie pokračuje analogicky. Je zrejmé, že ak prst ukazuje na koreň,
kľúč bude patriť do jeho podstromu.

% Otázka patričnosti do podstromu sa dá pre krajné prípady optimalizovať. Ak totiž vrchol, na ktorý
% ukazujeme, nemá napríklad pravého suseda, je zrejmé, že vačší kľúč ako je najvačší v tomto vrchole
% v strome nie je. Preto ľubovoľný kľúč do podstromu tohto vrchola patrí práve vtedy, keď je väčší
% ako jeho najmenší prvok. Analogicky to platí, ak chýba ľavý sused.
% 
% Operácia $\ins$ najprv pomocou operácie $\find$ nájde miesto, kam by mal vkladaný kľúč patriť. 
% Ak taký kľúč už v strome je, ďalší sa nevloží. V prípade, že sme vložili nový kľúč, môže sa stať, 
% že vrchol "pretečie", tzn.\ má viac ako 3 prvky%
% (pozri obr.~\ref{img:finger-insert}). Situácia sa vyrieši rovnako ako v \Bp-strome. 

Operácia $\ins$ najprv pomocou operácie $\find$ zistí, či štruktúra daný kľúč obsahuje. Ak nie, 
je zrejmé, že patrí práve do vrchola, kde $\find$ skončil. V prípade, že sme vložili nový kľúč, 
môže sa stať, že vrchol "pretečie", tzn.\ má viac ako 3 prvky%
(pozri obr.~\ref{img:finger-insert}). Ak vrchol pretiekol, použije sa COPY-UP 
a rozdelí sa na dve časti. Stredný jednoprvkový vrchol sa vloží do otca. Ak pretiekol, použijeme 
rovnaký postup, akurát s PUSH-UP. Pokračujeme analogicky, pokým štruktúra nebude znova 
v konzistentom stave.

Operácia $\delete$ najprv pomocou operácie $\find$ nájde miesto, kam by mal hľadaný kľúč patriť. 
Ak tam nie je, vymazávanie sa končí. Inak sa vymaže. Môže sa stať, že vrchol "podtečie", tzn.\ 
nemá kľúč. Tento problém sa rieši rovnako ako v \Bp-strome. 
Samozrejme je treba ošetriť konzistentnosť vo vyšších vrstvách, aby sa tam nenachádzali kópie 
kľúčov, ktoré už boli zo štruktúry vymazané (pozri obr. ~\ref{img:finger-delete}).  

\begin{figure}
\includegraphics[width=\columnwidth]{obrazky/finger-delete.png}
\caption{\emph{Vymazávanie kľúča $762$.} Odstrániť sa musí aj kópia vo vyšších vrstvách.}
\label{img:finger-delete}
\end{figure}

\begin{figure}
\includegraphics[width=\columnwidth]{obrazky/finger-insert.png}
\caption{\emph{Pretečenie vrchola.} Do stromu sme vložili prvok $308$ a 1) listový vrchol pretiekol. 
2) vrchol sa delí a kľúč sa kopíruje, aby originál zostal v liste 3) nový otec sa vkladá do vyššej vrstvy 
4) aj tento vrchol preteká, delí sa a vzniká nový koreň 5) finálny tvar.}
\label{img:finger-insert}
\end{figure}

\paragraph{Časová zložitosť.}
Keďže každý vrchol má aspoň dvoch synov, 2-3-4 strom má hĺbku $O(\log n)$, kde $n$ je počet kľúčov, 
a teda podporuje vykonávanie operácií v čase $O(\log n)$. Ak sa však použije prst, časová zložitosť 
vychádza na $O(\log d)$, kde $d$ je vzdialenosť pozície prsta a vrcholu, kam patrí cieľový kľúč, 
amortizovane dokonca na $O(1)$ \citep{sahni}. 

% \paragraph{Použitie.}
% Prst ako smerník na prvok štruktúry, ktorý umožňuje efektívnejší prístup k okolitým kľúčom, prvýkrát
% spomenuli Guibas et al. Vo svojej publikácii prezentujú B-strom podporujúci vyhľadávanie v $O(\log n)$
% a update-y dokonca v $O(1)$ čase, predpokladajúc, že je udržiavaných len $O(1)$ pohyblivých prstov \citet{sahni}.
% Pohyb prsta o $d$ pozícií trvalo $O(\log n)$ času. Na základe tejto práce navrhli Huddleston a Mehlhorn
% svoj vrstvovo spájaný 2-3-4 strom, ktorý bol neskôr upravený vďaka Belloch et al. na priestorovo efektívnejšiu
% alternatívu. Toto riešenie využíva jeden prst, s ktorým štruktúra ponúka rovnakú operačnú zložitosť ako 2-3-4 stromy.
% %Model bol dokonca zovšeobecnený na ($a$,$b$)-stromy, kde $b\geq 2a$.
% %Je zaujímavé, že pre 2-3 strom bola nájdená postupnosť vkladaní a vymazávaní, ktorá vyžaduje $\Omega(n\log n)$ krokov \citet{sahni}.

\paragraph{Vizualizácia.}
Strom s prstom je vizualizovaný pomocou \Bp-stromu s rádom 4, keďže jeho podmienky pre počet potomkov vyhovujú
danej štruktúre.
% Prst je samostaný pohyblivý článok, ktorý si pamätá iba vrchol, na ktorý ukazuje. Po stromovitej
% štruktúre sa vie hýbať vďaka informáciám získaným z daného vrcholu.
\def\find{$\mathop{find}(k)$}

\subsection{Strom s reverzami}
\emph{Strom s reverzami} je dátová štruktúra na uchovávanie permutácií. 
Poskytuje operácie 
\begin{itemize}
\item $\mathop{\mathit{insert}}(k)$ -- pridá do stromu $k$;
\item $\mathop{\mathit{find}}(k)$ -- zistí, ktorý prvok je na $k$-tom mieste permutácie a
\item $\mathop{\mathit{reverse}}(i,j)$ -- reverzne interval od $i$ po $j$.
\end{itemize}

\paragraph{Popis.}
Permutáciu reprezentujeme ako strom, v ktorom je \emph{inorder} poradie prvkov totožné 
s poradím prvkov v permutácií. Strom s reverzami môžeme implementovať pomocou ľubovoľného 
vyváženého stromu, ktorý podporuje rozdelenie a zreťazenie dvoch stromov v logaritmickom čase. 
My sme zvolili \emph{splay strom} pre jeho jednoduchosť. 

% Splay strom je štruktúrou binárny strom,
% líši sa od neho iba operáciami. Keď pracuje s ľubovoľným prvkom, na konci operácie bude vo vrchole
% buď daný kľúč alebo najbližší z jeho okoli. Na rozdiel od splay stromu, strom s reverzami nepracuje
% s kľúčmi, ale s poradím prvkov. Preto je nutné, aby mal

Aby sme vedeli efektívne vyhľadať $k$-ty prvok, budeme si pre každý vrchol udržiavať veľkosť jeho
podstromu. V operácií $\mathop{\mathit{find}}(k)$ sa vieme podľa toho rozhodnúť, či sa $k$-ty prvok nachádza v ľavom podstrome,
resp.~koľký prvok je v pravom podstrome. Po nájdení sa prvok presunie do koreňa pomocou operácie splay.

Operáciu \reverse\ implementujeme lenivo: 
strom najskôr rozdelíme na tri časti: $T_1,T_2,T_3$, pričom $T_2$ obsahuje interval od $i$-teho 
po $j$-ty prvok, $T_1$ obsahuje začiatok a $T_3$ koniec permutácie (obr.~\ref{img:rev2}). 
Koreň $T_2$ jednoducho označíme vlajkou, ktorá bude signalizovať, že podstrom je reverznutý a 
prvky sú v skutočnosti v opačnom poradí ako doteraz. Ak už koreň vlajku obsahuje, odstránime ju. 
Následne stromy $T_1,T_2,T_3$ opäť spojíme.


\begin{figure}
\includegraphics[width=\columnwidth]{obrazky/rev3trees.png}
\caption{\emph{Tri stromy.} Pri operácii \reverse\ sa strom rozdelí na 3 stromy. 
Vľavo prvky pred intervalom, vpravo prvky za ním. 
Na reverznutie intervalu stačí dať vlajku koreňu stredného stromu.}
\label{img:rev2}
\end{figure}

Pri takomto riešení musíme ešte upraviť vyhľadávanie a rotácie, aby brali do úvahy vlajky vo vrcholoch.
Najelegantnejšie riešenie je odstrániť vlajku vždy, keď na ňu narazíme:
Danému vrcholu odstránime vlajku, vymeníme mu synov a každému synovi vlajkový bit znegujeme.

Všetky operácie vieme implementovať v rovnakom čase ako operácie v splay tree, teda amortizovaná 
časová zložitosť oboch operácií je $O(\log n)$.

\paragraph{Použitie.}
Stromy s reverzami (pôvodne založené na AVL stromoch) navrhli \citet{chrobak}
na efektívnu implementáciu 2-opt heuristiky na riešenie problému obchodného cestujúceho.
Pri 2-opt heuristike sa snažíme reverzovať úseky cesty, kým nenájdeme lokálne minimum.

V bioinformatike sa tieto stromy používajú na triedenie orientovaných permuácií
pomocou reverzov \citep{reversals,reversals2}.

Za povšimnutie stojí fakt, že táto dátová štruktúra podporuje výmenu ľubovoľných dvoch blokov
v logaritmickom čase, keďže túto operáciu vieme odsimulovať pomocou štyroch reverzov.

\paragraph{Vizualizácia.}
Pre lepšiu vizualizáciu sme pridali do stromu nultý a posledný prvok. Tieto prvky
do reverzovateľného intervalu nepatria, majú však zmysel v prípade, ak sa reverzuje
interval, ktorý zahŕňa aspoň jeden okraj. V tom prípade v operácii \reverse\ nezostane
ani $T_1$ ani $T_3$ prázdny. Aby nevznikli problémy s operáciami, za krajné kľúče boli
zvolené hodnoty $0$ a číslo o jedna väčšie od aktuálneho maxima. Zároveň, pre lepšiu
predstavu, bolo pridané pole, v ktorom užívateľ vidí skutočné poradie prvkov, ktoré
zo stromu nie je až tak zjavné (obr.~\ref{img:rev1}). Pole simuluje operácie spolu so stromom, ale tie sú
na ňom vykonávané v lineárnom čase.

\begin{figure}
\includegraphics[width=\columnwidth]{obrazky/reversal.png}
\caption{\emph{Strom s reverzami.} Pre ľudí je názornejšie pole, počítaču viac vyhovuje strom.}
\label{img:rev1}
\end{figure}

\section{Haldy}
\subsection{d-nárna halda}
\subsection{Ľavicová halda}
\subsection{Skew halda}
\subsection{Párovacia halda}
Katka, zase spíš?!
[citacie]
\def\uf{Union-Find}
\def\null{\texttt{NULL}}
\def\makeset{$\mathop{\mathbf{makeset}}(x)$}
\def\find{$\mathop{\mathbf{find}}(x)$}
\def\union{$\mathop{\mathbf{union}}(x, y)$}

\section{\uf}
Sú problémy, ktoré vyžadujú spájanie objektov do množín a množín navzájom 
a následné určovanie, do ktorej množiny objekt patrí. Od takejto \emph{
dátovej štruktúry pre disjunktné množiny} očakávame, že si bude udržiavať 
jednoznačného \emph{zástupcu} každej množiny a bude poskytovať 
tieto tri oprácie: 
\begin{itemize}
\item \makeset\ -- vytvorí novú množinu s jedným prvkom, ktorý 
nepatrí do žiadnej inej množiny;
\item \find\ -- nájde zástupcu množiny, v ktorej sa 
prvok $x$ nachádza;
\item \union\ -- vytvorí novú množinu, ktorá obsahuje 
všetky prvky v množinách, ktorých zástupcovia sú $x$ a $y$. Tieto 
množiny zmaže. Ďalej vyberie nového zástupcu novej množiny. Pre 
jednoduchosť, táto operácia predpokladá, že $x$ a $y$ sú 
zástupcovia množín.
\end{itemize}
Vďaka dvom hlavným operáciam \find\ a \union\ 
je táto dátová štruktúra známejšia pod pojmom \emph{\uf}, ktorý 
používame aj my. Medzi najznámejšie problémy, ktoré sa riešia pomocou 
\uf\ patria Kruskalov algoritmus na nájdenie najlacnejšej kostry 
\citep{kruskal} a unifikácia \citep{unif}. Veľmi triviálne použitie je 
zistenie počtu komponentov v grafe. Existujú aj iné problémy z teórie 
grafov, napr. \citet{paths1}.

Vďaka častej asociácií objektov a spájania množín ako vrcholy a hrany grafu 
sa často dátová štruktúra abstraktne reprezentuje ako 
\emph{les} -- množina zakorenených stromov. 
Konkrétnou implementáciou potom býva pole objektov --- vrcholov. Ku každému 
objektu sa musí udržiavať smerník $p(x)$ na otca v strome. Smerník zástupcu 
množiny zvyčajne ukazuje na seba ($p(x) = x$). V našej implementácií však 
smerník zástupcu množiny ukazuje na hodnotu \null.

Operácia \makeset\ teda vytvorí nový prvok $x$ a nastaví $p(x) = \null$. 
Operáciu \find\ vykonáme tak, že budeme sledovať cestu po smerníkoch, až 
kým nenájdeme zástupcu. 
Operáciu \union\ ide najjednoduchšie vykonať tak, že presmerujeme smerník 
$p(y)$ na prvok $x$, teda $p(y) = x$. 
Môžeme ľahko pozorovať, že takýto \emph{naivný} spôsob je neefektívny, 
lebo nám operácia \find\ v najhoršom prípade, na $n$ prvkoch, trvá $O(n)$ 
krokov. 

Existujú dva prístupy ako zlepšiť operácie a tým aj zrýchliť ich vykonanie. 
Sú to: heuristika \emph{union podľa ranku} a rôzne heuristiky na 
\emph{kompresiu cesty}. Prvá heuristika pridáva ku algoritmom hodnotu 
$rank(x)$, ktorá bude určovať najväčšiu možnú hĺbku podstromu zakorenenú 
vrcholom $x$. V tom prípade pri o\-pe\-rá\-cií \makeset\ zadefinujeme 
$rank(x) = 0$. 
Pri o\-pe\-rá\-cií \union\ vždy porovnáme $rank(x)$ a $rank(y)$, aby sme zistili, 
ktorý zástupca predstavuje menší strom. Smerník tohto zástupcu potom napojíme 
na zástupcu s výšším rankom. Zástupca novej množiny bude ten s vyšším rankom. 
Ak sú oba ranky rovnaké, vyberieme ľubovoľného zo zástupcov $x$ a $y$, 
jeho rank zvýšime o jeden a smerník ostatného zástupcu bude ukazovať 
na tohto zástupcu. Zástupcom novej množiny bude vybratý zástupca. 

Druhou heuristikou je kompresia cesty. Algoritmov na efektívnu kompresiu 
cesty je veľa \citep{paths2}. Tu popíšeme tie najefektívnejšie. Prvou z nich 
je \emph{jednoduchá kompresia cesty} \citep{comp1}. Pri vykonávaní 
operácie \find, po tom, 
ako nájdeme zástupcu množiny obsahujúcej prvok $x$, smerníky prvkov 
navštívených po ceste (včetne $x$) presmerujeme na zástupcu množiny. Toto 
síce spomalí prvé vykonávanie, ale výrazne zrýchli ďalšie hľadania. 
Druhou heuristikou je \emph{delenie cesty} \citep{comp2}. Pri vykonávaní 
operácie \find\ 
pripojíme každý vrchol\footnote{okrem koreňa a synov koreňa, 
keďže tie deda a otca resp. deda nemajú} v ceste od vrcholu $x$ po koreň stromu 
na otca jeho otca. 
Treťou heuristikou je \emph{pólenie cesty} \citep{comp2}. Pri vykonávaní 
operácie \find\ 
pripojíme každý druhý vrchol\footnote{okrem koreňa a synov koreňa, 
keďže tie deda a otca resp. deda nemajú} 
v ceste od vrcholu $x$ po koreň stromu na otca jeho otca. 

Podľa \citet{galil} je najefektívnejšie pólenie cesty pred delením, ktoré 
spotrebuje zhruba dva krát viac smerníkov a jednoduchou kompresiou, ktorá 
vyžaduje dva behy.

%citovat Walkerov alg. - to mám robiť tu?



\def\k{k}
%\def\kluc{kľúč}
\def\put{$\mathop{insert}(\k)$}
\def\find{$\mathop{find}(\k)$}
\def\delete{$\mathop{delete}(\k)$}
\def\trie{trie}
\def\uz{{\tt\$}}

\section{Písmenkový strom}
\emph{Abeceda} je množina \emph{znakov}. \emph{Slovo} je postupnosť znakov 
z danej abecedy. \emph{Písmenkový strom} je \emph{zakorenený strom}, v ktorom 
každá hrana obsahuje práve jeden znak z abecedy alebo \emph{ukončovací znak}. 
\emph{Ukončovací znak} je ľubovoľný, dopredu dohodnutý symbol, ktorý sa 
v abecede nenachádza. My používame ako abecedu veľké znaky anglickej 
abecedy a ukončovací znak značíme dolárom (\uz).



\paragraph{Popis.}
Strom obsahuje slová nazývané \emph{kľúče}. Po prejdení cesty 
z koreňa do vrcholu s \emph{ukončovacím znakom} prečítame kľúč.%
\footnote{Na hranách $h_1, h_2, \ldots, h_n$ sú znaky $z_1, z_2, \ldots, 
z_{n-1}, $\uz, ktoré po zreťazení utvoria slovo $z_1z_2\ldots{z_{n-1}}$.} 
Teda, oproti binárnym vyhľadávacím stromom je hlavný rozdiel v tom, že 
kľúče nie sú uložené vo vrcholoch, ale samotná poloha v strome určuje kľúč. 
Písmenkový strom sa podobne ako binárny vyhľadávací strom využíva ako 
\emph{asociatívne pole (slovník)}, takže poskytuje tieto tri operácie:
\begin{itemize}
\item $\mathop{\mathbf{insert}}(\k)$ -- pridá do stromu kľúč $\k$;
\item $\mathop{\mathbf{find}}(\k)$ -- zistí, či sa v strom kľúč $\k$ nachádza;
\item $\mathop{\mathbf{delete}}(\k)$ -- odstráni zo stromu kľúč $\k$ a 
prípadne vyrieši zmeny v strome.
\end{itemize}
Všetky operácie začínajú v koreni a ku kľúču pridávajú ukončovací znak, 
teda pracujú s reťazcom $\k\emph{\$}$. 

Operácia \put\ vloží do stromu vstupný reťazec tak, že z reťazca berie znaky 
a prechádza po príslučných hranách. Ak hrana so znakom neexistuje pridá ju. 

Operácia \find\ sa spustí z koreňa podľa postupnosti znakov. Ak hrana, 
po ktorej sa má spustiť neexistuje, daný kľúč sa v strome nenachádza. 
Ak prečítame celý vstupný reťazec, daný kľúč sa v strome nachádza.

Operácia \delete\ najprv pomocou operácie \find\ zistí umiestnenie kľúča. 
Ak sa kľúč v strome nachádza, algoritmus odstráni hranu s ukončovacím 
symbolom a vrchol, ktorý bol na nej zavesený. V tomto štádiu sa nám môže 
stať, že v strome ostane takzvaná \emph{mŕtva vetva} -- nie je ukončená 
ukončovacím znakom. Pre fungovanie stromu to nevadí, všetky operácie by 
prebiehali správne, ale takto štruktúra zaberá zbytočne veľa miesta. 
Preto je dobré túto mŕtvu vetvu odstrániť.

\begin{figure}
\includegraphics[width=\columnwidth]{obrazky/trieinsertsmall.png}
\caption{\emph{Operácia \put.} Vo vizualizácií naznačujeme, kade sa 
presunieme.} 
\label{img:trieinsert} 
\end{figure}

\begin{figure}
\includegraphics[width=\columnwidth]{obrazky/triedeletesmall.png}
\caption{\emph{Operácia \delete.} \emph{Mŕtva vetva} je vyznačená červenou.
%Najprv zistíme, či je kľúč v strome, 
%potom ho zmažeme a potom zmažeme mŕtvu vetvu, ktorá je vyznačená 
%červenou farbou.
} 
\label{img:triedelete} 
\end{figure}

\paragraph{Použitie.}
Vďaka tomu, že písmenkový strom udržiava spoločné \emph{prefixy} kľúčov sa 
nazýva aj \emph{prefixový strom}.
Prvý krát popísal písmenkový strom \citet{fredkin}, ktorý používal názov 
\emph{trie memory}, keďže išlo o spôsob udržiavania dát v pamäti. Pojem 
\emph{trie}\footnote{Z anglického re\emph{trie}val -- získanie.} 
sa rozšíril a používa sa celosvetovo.
%používal operácie 
%$\mathop{storage}$ (\put), $\mathop{retrieval}$ (\find) 
%a $\mathop{deletion}$ (\delete) a dátovú štruktúru nazýval \emph{trie 
%memory}, keďže išlo naozaj o spôsob uloženia dát v pamäti.

O niečo neskôr \citet{knuth} uviedol vo svojej knihe ako príklad na 
písmenkový strom vreckový slovník. V tom istom diele uviedol aj možnosť 
komprimovania vetiev a možnosť prerobenia $m$-árneho \trie\ na binárny. 
\citet{knuth} však ukázal len komprimovanie koncov vetiev. Písmenkový 
strom, v ktorom každý vrchol, ktorého otec má len jedného syna 
je zlúčený s otcom\footnote{Na hranách teda nie sú znaky, ale slová.}, 
popísal \citet{patricia} a zaviedol pre ňho pojem \emph{PATRICIA}. 
Pre túto dátovú štruktúru sa používa aj pojem \emph{radix tree (radix trie)}. 

Dátovú štruktúru podobnú \emph{hashovacej tabuľke} a písmenkovému stromu 
popísal vo svojej dizertačnej práci \citet{liang} a nazval ju \emph{packed 
trie}. Oproti obyčajnému písmenkovému stromu výrazne šetrila miesto. 
V práci ju využil na vytvorenie vzorcov pre slabikovanie slov.

Písmenkové stromy sa podobajú na \emph{konečné automaty}. 
Vznikli rôzne modifikácie stromov na automaty, ktorých hlavnou výhodou je, 
že v komprimovanej podobe spájajú nielen predpony, ale aj prípony slov 
a teda v slovách ľudských jazykov výrazne znižujú pamäťový priestor potrebný 
na uchovanie dátovej štruktúry \citep{scrabble,ca}. 

Priamočiare je použitie písmenkového stromu na utriedenie poľa slov. 
Všetky slová sa pridajú do stromu a potom sa spraví \emph{preorderový prechod} 
stromu. Túto myšlienku spracovali \citet{burstsort1} a veľmi výrazne zrýchlil 
triedenie dlhých zoznamov slov. Neskôr tento algoritmus vylepšili 
\citet{burstsort2}. Kvôli tomu, ako algoritmus pracuje, 
sa nazýva \emph{burstsort}.

Špeciálnym použitím písmenkového stromu je vytvorenie stromu zo všetkých 
prípon slova. Táto dátova štruktúra sa nazýva \emph{sufixový strom} a dá sa 
mo\-di\-fi\-ko\-vať na udržiavanie viacerých slov. Tieto štruktúry majú 
veľmi veľa praktických využití \citep{gusfield}. 

%Friker, neobzeraj baby a pracuj! 
%Nejaké citovateľné práce o trie?


\section{Záver}
work in progress; co sme spravili, preco sme lepsi, co este chceme/treba spravit, co je rozrobene
Paly?

\subsection{Príspevky autorov}
Katka Kotrlová obohatila projekt o vizualizácie d-nárnej, ľavicovej, skew a
párovacej haldy, Viktor Tomkovič pridal vizualizácie union-findu a písmenkového
stromu, Tatiana Tóthová vizualizovala B$^+$-strom, strom s prstom a strom s
reverzami a Pavol Lukča dorobil históriu
krokov a operácií do takmer všetkých slovníkov a venoval sa refaktorovaniu
zdrojového kódu. Na príprave tohto textu sa podieľali všetci autori.


\section*{Poďakovanie}
Autori by sa chceli poďakovať školiteľovi za veľa dobrých rád 
a odborné vedenie pri práci.



\nocite{*}
\bibliographystyle{apalike}
\bibliography{references}

%% citacie ulozte do suboru references.bib
%% na populaciu zoznamu literatury pouzite program
%%
%% bibtex references
%%
%% po ktorom je potrebne dokument znova zlatexovat

\end{document}
