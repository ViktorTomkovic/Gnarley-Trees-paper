\section{Union Find}
Sú problémy, ktoré vyžadujú spájanie objektov do množín a množín navzájom 
a následné určovanie, do ktorej množiny objekt patrí. Od takejto \emph{
dátovej štruktúry pre disjunktné množiny} očakávame, že si bude udržiavať 
jednoznačného \emph{zástupcu} každej množiny a bude poskytovať 
tieto tri oprácie: 
\begin{itemize}
\item \textbf{makeset} -- vytvorí novú množinu s jedným prvkom, ktorý 
nepatrí do žiadnej inej množiny;
\item \textbf{find($x$)} -- nájde zástupcu množiny, v ktorej sa 
prvok $x$ nachádza;
\item \textbf{union($x$, $y$)} -- vytvorí novú množinu, ktorá obsahuje 
všetky prvky v množinách, ktorých zástupcovia sú $x$ a $y$. Tieto 
množiny zmaže. Ďalej vyberie nového zástupcu novej množiny. Pre 
jednoduchosť, táto operácia predpokladá, že $x$ a $y$ sú 
\emph{zástupcovia} množín.
\end{itemize}
Vďaka dvom hlavným operáciam \textbf{find} a \textbf{union} sa na Slovensku 
je táto dátová štruktúra známejšia pod poj\-mom \emph{Union-Find}, ktorý 
používame aj my. 

Na reprezentovanie dátovej štruktúry sme si zvolili štruktúru podobnú 
lesu: každá množina je reprezentovaná stromom a zástupca množiny je koreň. 
Ďalej je dôležité, že ku objektom vieme pristupovať pria\-mo (pole).
Teda, objekty majú svoje \emph{id} a pointer \emph{p($x$)} 
ukazujúci na otca v strome. Pointer zástupcu množiny ukazuje na 
hodnotu \texttt{NULL}.\\
rozne kopresie cesty
citovat Walkerov alg.
\\
Yahoo! Walker, Texas ranger, FTW!