\def\Bp{$\hbox{\rm B}^+$}

\section{Vyvážené stromy}

Pôvodná verzia \citep{kuko} obsahovala vizualizáciu viacerých vyvážených stromov.
K nim sme pridali \Bp-stromy, stromy s prstom a stromy s reverzami.

\subsection{B$^+$-strom}

\paragraph{Popis.}
\emph{\Bp-strom} je variácia B-stromu, v ktorom sú všetky kľúče uložené v listoch
a listy sú pospájané do spájaného zoznamu. Prvky vo vnútorných vrcholoch slúžia len
na navigáciu. 

\Bp-strom rádu $b$ je strom, v ktorom má každý vnútorný vrchol najviac $b$
a najmenej $\lfloor b/2 \rfloor$ synov (okrem koreňa, ktorý má najmenej dvoch synov).
Vďaka tomu je dobre vyvážený a jeho operácie sú vykonávané v logaritmickom čase.
\Bp-strom je \emph{asociatívne pole (slovník)}, čiže poskytuje tieto tri operácie:
\begin{itemize}
\item $\ins(x)$ -- pridá do stromu prvok $x$;
\item $\find(x)$ -- zistí, či sa $x$ v strome nachádza;
\item $\delete(x)$ -- odstráni $x$ zo stromu.
\end{itemize}

Operácia $\find(x)$ začne v koreni, nájde v ňom prvý kľúč väčší od hľadaného.
Nech je $i$-ty v poradí, potom hľadanie pokračuje $i$-tou vetvou.
(Ak je $x$ väčšie ako všetky kľúče, pokračujeme poslednou vetvou.)
V liste už len skontrolujeme, či sa v ňom hľadaný kľúč nachádza.

% Definujeme dve operácie: {\sc Copy-Up} a {\sc Push-Up}, ktoré používa operácia $\ins$.
% Ak má vrchol viac prvkov, ako je maximálny limit, treba ho zmenšiť. Rozdelí sa na dve časti.
% Ak vrchol nie je listom, použije sa {\sc Push-Up}, najmenší kľúč pravej časti sa vyberie
% a stane sa otcom vytvorených dvoch častí. Pokiaľ to list je, kľúč v ňom musí zostať,
% preto sa iba skopíruje. Táto operácia sa nazýva {\sc Copy-Up}.

Operácia $\ins(x)$ najprv pomocou operácie $\find$ zistí, či štruktúra daný kľúč už obsahuje.
Ak nie, je zrejmé, že $x$ patrí práve na miesto, kde $\find$ skončil.
Môže sa stať, že vrchol po vložení "pretečie" -- bude obsahovať $b+1$ prvkov. V takom prípade
ak má vrchol brata s menej ako $b$ prvkami, môžeme jeden kľúč presunúť k nemu.
Ak sú susedné vrcholy plné, vrchol rozdelíme na dva, pričom stredný kľúč skopírujeme
k otcovi. (Ten sa môže tiež preplniť, čím vznikne kaskáda rozdelení, ktorá skončí
v najhoršom prípade v koreni.)
% Nový vrchol s jedným kľúčom, ktorý vznikol, vložíme do
% otcovského vrcholu. Ak otcovský vrchol presiahol najväčšiu možnú veľkosť, znova sa aplikuje
% popísaný algoritmus s jedným rozdielom -- namiesto {\sc Copy-Up} sa použije {\sc Push-Up}.

Podobne v operácií $\delete$ môže vrchol "podtiecť". Ak má suseda s aspoň $\lfloor b/2 \rfloor+1$
kľúčmi, môže si jeden požičať od neho. V opačnom prípade môžeme vrchol s jeho susedom zlúčiť.
% najprv pomocou $\find$ nájde kľúč, potom ho z vrcholu odstráni. Tento vrchol
% môže mať po odstránení menší počet kľúčov ako minimálny limit. Vtedy, ak sa dá, sa prenesie
% jeden kľúč zo súrodenca. Ak sa nedá, vrchol sa s ním zlúči. Zároveň sa k nim pridá aj kľúč
% z otcovského vrcholu, ktorý ich rozdeľoval. Pokiaľ to spôsobilo, že otcovský vrchol má menej
% kľúčov, ako je povolené, znova sa aplikuje predošlý algoritmus. Keďže na koreň sa nevzťahuje
% minimálny limit, po skončení bude strom zaručene v konzistentnom tvare.

\paragraph{Časová zložitosť a použitie.}
\Bp-stromy sú vhodnou dátovou štruktúrou pre dáta uložené na disku: keďže dĺžka prístupu na
disk je v porovnaní s výpočtami v hlavnej pamäti veľmi veľká, snažíme sa ich počet minimalizovať.
Hoci časová zložitosť všetkých operácií je $O(b\log_b n)$, potrebujeme iba $O(\log_b n)$ prístupov
na disk. Ak zvolíme vhodný rád, vieme jednotlivé vrcholy dobre napasovať na stránky a tým regulovať
ako počet prístupov k pamäti, tak jej zaplnenie.

Hlavné využitie \Bp-stromov je v databázových systémoch. Stromy vieme rozšíriť tak, aby podporovali
rôzne agregačné funkcie, ako napríklad súčet, minimum, či priemer daného intervalu pomocou $O(\log_b n)$
prístupov na disk. Vypísať všetky prvky z daného intervalu dokážeme pomocou $O(\log_b n + t/b)$ prístupov na disk.
% \Bp-strom podporuje efektívne vyhľadanie prvkov poľa, ktoré patria do daného intervalu.
% Algoritmus nájde jeden krajný bod a vďaka spájanému zoznamu, vytvorenému z listov, ostatné prvky
% postupne prečíta. Zložitosť je $O(\log_b(n) + t/b)$, kde $t$ je počet výsledných kľúčov z hľadaného intervalu.

Ďaľšia výhoda \Bp-stromov oproti B-stromom sa prejaví, ak máme utriedený zoznam dát a chceme z neho vytvoriť \Bp-strom:
\Bp-strom môžeme vystavať odspodu. Takýto postup vyžaduje $O((n/b)\cdot\log_b n)$ prístupov na disk,
čo je $b$-krát rýchlejšie ako spraviť $n$ volaní $\ins$.




\subsection{Strom s prstom}
Tradičné vyvážené vyhľadávacie stromy podporujú vyhľadávanie v čase $O(\log n)$.
Prst je smerník na konkrétny vrchol, ktorý umožňuje efektívnejší prístup k vrcholom
v jeho blízkom okolí. Hovoríme, že vyhľadávací strom podporuje vyhľadávanie s prstom
(tzv.\ \emph{finger search tree}), ak kľúč vo vzdialenosti $d$ dokážeme nájsť v čase $O(\log d)$.
Špeciálne predchodcu a následníka vieme nájsť v konštantnom čase.

Existuje viacero riešení, ktoré podporujú vyhľadávanie s prstom, v našom programe
sme implementovali upravený 2-3-4$^+$ strom \citep{finger}.

\paragraph{Popis.}
2-3-4$^+$ strom je \Bp-strom rádu 4, t.j.\ kľúče sú uložené v listoch a vnútorné vrcholy
majú stupeň 2, 3 alebo 4. Pre podporu vyhľadávania s prstom spojíme všetky vrcholy
na rovnakej úrovni (v rovnakej vzdialenosti od koreňa) do obojsmerného spájaného zoznamu.
Ak sú nejaké dva vrcholy spojené takouto hranou, budeme hovoriť, že sú susedia.

% Prst, ako už bolo spomenuté, ukazuje na nejaký vrchol. Môže sa pohybovať po všetkých hranách
% a pomocou neho sa vykonávajú všetky operácie. Keďže sú všetky kľúče uložené v listoch, prst
% na tejto vrstve začína, aj končí.

Operácia $\find$ začína v liste, na mieste, kam ukazuje prst. Vyhľadávanie pozostáva z dvoch
fáz: V prvej fáze postupujeme nahor, až kým hľadaný kľúč nepatrí do nášho alebo susedovho podstromu.
(V prípade potreby použijeme úrovňovú hranu, ktorou sa dostaneme do susedného vrcholu.)
V druhej fáze potom zostupujeme nadol ako pri štandardnom vyhľadávaní.
Pri $\ins$ a $\delete$ najskôr pomocou prstu nájdeme vhodné miesto a následne kľúč pridáme/vymažeme,
rovnako ako v \Bp-strome. 

Takto implementované vyhľadávanie trvá $O(\log d)$, vkladanie a vymazávanie (po tom ako sme vrchol
našli) má konštantnú amortizovanú zložitosť.


% Skontroluje, či by kľúč mal patriť do daného
% vrcholu. Ak nie, pozrie sa, či nepatrí do niektorého zo susedov. Ak áno, prst sa tam presunie
% a vyhľadávanie sa skončilo. Inak smerník prejde o úroveň vyššie, na otcovský vrchol. Ak hľadaný
% kľúč patrí do jeho podstromu (t.j.\ je väčší ako jeho najmenší kľúč a menší ako ten najväčší),
% zíde po hranách do listu, kde by sa daný kľúč mal nachádzať. Keď do podstromu nepatrí, skontroluje,
% či nepatrí do podstromu susedov. Ak áno, prejde po vrstevnej hrane na suseda a následne zíde až
% do listu, kde by mal kľúč byť. Pokiaľ prst nenarazil na správny podstrom, znova sa presunie smerom
% nahor po otcovskej hrane. Hľadanie pokračuje analogicky.
%Je zrejmé, že ak prst ukazuje na koreň, kľúč bude patriť do jeho podstromu.

% Otázka patričnosti do podstromu sa dá pre krajné prípady optimalizovať. Ak totiž vrchol, na ktorý
% ukazujeme, nemá napríklad pravého suseda, je zrejmé, že vačší kľúč ako je najvačší v tomto vrchole
% v strome nie je. Preto ľubovoľný kľúč do podstromu tohto vrcholu patrí práve vtedy, keď je väčší
% ako jeho najmenší prvok. Analogicky to platí, ak chýba ľavý sused.
% 
% Operácia $\ins$ najprv pomocou operácie $\find$ nájde miesto, kam by mal vkladaný kľúč patriť. 
% Ak taký kľúč už v strome je, ďalší sa nevloží. V prípade, že sme vložili nový kľúč, môže sa stať, 
% že vrchol "pretečie", tzn.\ má viac ako 3 prvky%
% (pozri obr.~\ref{img:finger-insert}). Situácia sa vyrieši rovnako ako v \Bp-strome. 

% Operácia $\ins$ najprv pomocou operácie $\find$ zistí, či štruktúra daný kľúč obsahuje. Ak nie, 
% je zrejmé, že patrí práve do vrcholu, kde $\find$ skončil. V prípade, že sme vložili nový kľúč, 
% môže sa stať, že vrchol "pretečie", tzn.\ má viac ako 3 prvky %
% (pozri obr.~\ref{img:finger-insert}). Ak vrchol pretiekol, použije sa {\sc Copy-Up}
% a rozdelí sa na dve časti. Stredný jednoprvkový vrchol sa vloží do otca. Ak pretiekol, použijeme 
% rovnaký postup, akurát s {\sc Push-Up}. Pokračujeme analogicky, pokým štruktúra nebude znova 
% v konzistentom stave.

% \begin{figure}
% \includegraphics[width=\columnwidth]{obrazky/finger-insert.png}
% \caption{\emph{Pretečenie vrcholu.} Do stromu sme vložili prvok $308$ a 1) listový vrchol pretiekol. 
% 2) vrchol sa delí a kľúč sa kopíruje, aby originál zostal v liste 3) nový otec sa vkladá do vyššej vrstvy 
% 4) aj tento vrchol preteká, delí sa a vzniká nový koreň 5) finálny tvar.}
% \label{img:finger-insert}
% \end{figure}
% 
% Operácia $\delete$ najprv pomocou operácie $\find$ nájde miesto, kam by mal hľadaný kľúč patriť. 
% Ak tam nie je, vymazávanie sa končí. Inak sa vymaže. Môže sa stať, že vrchol "podtečie", tzn.\ 
% nemá kľúč. Tento problém sa rieši rovnako ako v \Bp-strome. 
% Samozrejme je treba ošetriť konzistentnosť vo vyšších vrstvách, aby sa tam nenachádzali kópie 
% kľúčov, ktoré už boli zo štruktúry vymazané (pozri obr.~\ref{img:finger-delete}).  
% 
% \begin{figure}
% \includegraphics[width=\columnwidth]{obrazky/finger-delete.png}
% \caption{\emph{Vymazávanie kľúča $762$.} Odstrániť sa musí aj kópia vo vyšších vrstvách.}
% \label{img:finger-delete}
% \end{figure}
% 
% \paragraph{Časová zložitosť.}
% Keďže každý vrchol má aspoň dvoch synov, 2-3-4 strom má hĺbku $O(\log n)$, kde $n$ je počet kľúčov, 
% a teda podporuje vykonávanie operácií v čase $O(\log n)$. Ak sa však použije prst, časová zložitosť 
% vychádza na $O(\log d)$, kde $d$ je vzdialenosť pozície prsta a vrcholu, kam patrí cieľový kľúč, 
% amortizovane dokonca na $O(1)$ \citep{sahni}.

% \paragraph{Použitie.}
% Prst ako smerník na prvok štruktúry, ktorý umožňuje efektívnejší prístup k okolitým kľúčom, prvýkrát
% spomenuli Guibas et al. Vo svojej publikácii prezentujú B-strom podporujúci vyhľadávanie v $O(\log n)$
% a update-y dokonca v $O(1)$ čase, predpokladajúc, že je udržiavaných len $O(1)$ pohyblivých prstov \citet{sahni}.
% Pohyb prsta o $d$ pozícií trvalo $O(\log n)$ času. Na základe tejto práce navrhli Huddleston a Mehlhorn
% svoj vrstvovo spájaný 2-3-4 strom, ktorý bol neskôr upravený vďaka Belloch et al. na priestorovo efektívnejšiu
% alternatívu. Toto riešenie využíva jeden prst, s ktorým štruktúra ponúka rovnakú operačnú zložitosť ako 2-3-4 stromy.
% %Model bol dokonca zovšeobecnený na ($a$,$b$)-stromy, kde $b\geq 2a$.
% %Je zaujímavé, že pre 2-3 strom bola nájdená postupnosť vkladaní a vymazávaní, ktorá vyžaduje $\Omega(n\log n)$ krokov \citet{sahni}.

% \paragraph{Vizualizácia.}
% Strom s prstom je vizualizovaný pomocou \Bp-stromu s rádom 4, keďže jeho podmienky pre počet potomkov vyhovujú
% danej štruktúre.
% Prst je samostaný pohyblivý článok, ktorý si pamätá iba vrchol, na ktorý ukazuje. Po stromovitej
% štruktúre sa vie hýbať vďaka informáciám získaným z daného vrcholu.

\subsection{Strom s reverzami}
\emph{Strom s reverzami} je dátová štruktúra na uchovávanie permutácií. 
Majme permutáciu $\pi$ na množine $\{1,2,\ldots,n\}$; dátová štruktúra
poskytuje operácie 
\begin{itemize}
%\item $\ins(k)$ -- pridá do stromu $k$;
\item $\reverse(i,j)$ -- preklopí poradie prvkov v intervale od $i$ po $j$,
\item $\find(k)$ -- zistí, ktorý prvok je na $k$-tom mieste permutácie $\pi$.
\end{itemize}

\paragraph{Popis.}
Permutáciu reprezentujeme ako strom, v ktorom je \emph{inorder} poradie prvkov totožné 
s poradím prvkov v permutácií. Strom s reverzami môžeme implementovať pomocou ľubovoľného 
vyváženého stromu, ktorý podporuje rozdelenie a zreťazenie dvoch stromov v logaritmickom čase. 
My sme zvolili \emph{splay strom} pre jeho jednoduchosť. 

% Splay strom je štruktúrou binárny strom,
% líši sa od neho iba operáciami. Keď pracuje s ľubovoľným prvkom, na konci operácie bude vo vrchole
% buď daný kľúč alebo najbližší z jeho okoli. Na rozdiel od splay stromu, strom s reverzami nepracuje
% s kľúčmi, ale s poradím prvkov. Preto je nutné, aby mal

Niektoré vrcholy môžu byť označené vlajkou, ktorá signalizuje, že celý podstrom je reverznutý a 
prvky sú v skutočnosti v opačnom poradí (pozri obr.~\ref{img:rev1}).

Operáciu $\reverse$ implementujeme lenivo: 
strom najskôr rozdelíme na tri časti: $T_1,T_2,T_3$, pričom $T_2$ obsahuje interval od $i$-teho 
po $j$-ty prvok, $T_1$ obsahuje začiatok a $T_3$ koniec permutácie (obr.~\ref{img:rev2}). 
Koreň $T_2$ jednoducho označíme vlajkou. Ak už koreň vlajku obsahuje, odstránime ju. 
Následne stromy $T_1,T_2,T_3$ opäť spojíme.

Aby sme vedeli efektívne vyhľadať $k$-ty prvok, budeme si pre každý vrchol udržiavať veľkosť jeho
podstromu. V operácií $\mathop{\mathit{find}}(k)$ sa vieme podľa toho rozhodnúť, či sa $k$-ty prvok nachádza v ľavom podstrome,
resp.~koľký prvok je v pravom podstrome. Po nájdení sa prvok presunie do koreňa pomocou operácie splay.

\begin{figure}
\includegraphics[width=\columnwidth]{obrazky/reversal.png}
\caption{\emph{Strom s reverzami.} Hore permutácia $\pi$, dolu reprezentácia pomocou splay stromu.
Vlajky vo vrcholoch 8 a 9 signalizujú, že úseky $5,8$ a $4,9,3,1,7$ sú v opačnom poradí. Číslo vľavo
hore od vrcholu je počet vrcholov v danom podstrome.}
\label{img:rev1}
\end{figure}

Pri takomto riešení musíme ešte upraviť vyhľadávanie a rotácie, aby brali do úvahy vlajky vo vrcholoch.
Najelegantnejšie riešenie je odstrániť vlajku vždy, keď na ňu narazíme:
Danému vrcholu odstránime vlajku, vymeníme mu synov a každému synovi vlajkový bit znegujeme.

Všetky operácie vieme implementovať v rovnakom čase ako operácie v splay tree, teda amortizovaná 
časová zložitosť oboch operácií je $O(\log n)$.

\paragraph{Použitie.}
Stromy s reverzami (pôvodne založené na AVL stromoch) navrhli \citet{chrobak}
na efektívnu implementáciu 2-opt heuristiky na riešenie problému obchodného cestujúceho.
Pri 2-opt heuristike sa snažíme preklápať rôzne úseky cesty, kým nenájdeme lokálne minimum.

V bioinformatike sa tieto stromy používajú na triedenie orientovaných permuácií
pomocou reverzov \citep{reversals,reversals2}.

Za povšimnutie stojí fakt, že táto dátová štruktúra podporuje výmenu ľubovoľných dvoch blokov
v logaritmickom čase, keďže túto operáciu vieme odsimulovať pomocou štyroch reverzov.

\paragraph{Vizualizácia.}
Pre lepšiu predstavu, bolo pridané pole, v ktorom užívateľ vidí skutočné poradie prvkov, ktoré
zo stromu nie je až tak zjavné (obr.~\ref{img:rev1}). Pole simuluje operácie spolu so stromom,
tie sa však vykonávajú v lineárnom čase.

Do stromu sme pridali ako zarážky nultý a posledný prvok. Tieto prvky do reverzovateľného
intervalu nepatria, majú však zmysel v prípade, ak sa reverzuje interval, ktorý zahŕňa
aspoň jeden okraj: V tom prípade v operácii $\reverse$ nezostane $T_1$ ani $T_3$ prázdny. 
%Aby nevznikli problémy s operáciami, za krajné kľúče boli zvolené hodnoty $0$ a číslo o jedna väčšie od aktuálneho maxima.

\begin{figure}
\includegraphics[width=\columnwidth]{obrazky/rev3trees.png}
\caption{Pri operácii \emph{reverse} sa strom rozdelí na tri časti. 
Vľavo prvky pred intervalom, vpravo prvky za ním. 
Na reverznutie intervalu stačí pridať vlajku vrcholu 3 (koreň stredného stromu).}
\label{img:rev2}
\end{figure}

