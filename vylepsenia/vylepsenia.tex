\section{Rozšírenie predošlej práce}

Projekt Gnarley Trees sme rozšírili nielen o vizualizácie ďalších dátových
štruktúr, ale pribudli aj softvérové (vizualizačné) vylepšenia.

%jednotlive vizualizacie a implementovane ficurie
%co sa zmenilo od bakalarky? zoomovanie, komentare, tree layouty, historia a nove ds

\todo{toto treba kompletne vymyslieť a napísať a potom tam dakde pichnúť:}

\paragraph{Tesnejšie vykresľovanie grafov.}
Stromy sa vykresľovali tak, že sa spravil obdĺžnikový obal podstromov daného 
vrcholu, tieto obdĺžniky sa usporiadali vedľa seba a k nim sa pripojil daný 
vrchol a okolo celej štruktúry sa spravil obdĺžnik. Toto je ale spôsob, ktorý 
nešetrí priestor a pri štruktúrach ako písmenkový strom by výsledné stromy 
vyzerali škaredo. Preto sme sa rozhodli pre stromy implementovať 
alternatívny spôsob rozloženia, ktorý vymyslel \citet{reingold} a pre $m$-árne 
stromy rozšíril \citet{walker}. Tieto rozloženia vykresľujú vrcholy stromov 
najtesnejšie, pričom dodržujú tieto pravidlá: vrcholy v rovnakej hĺbke musia 
byť vykreslené na jednej priamke a priamky určujúce jednotlivé úrovne majú byť 
rovnobežné; poradie synov má byť zachované; otec má ležať v strede svojich 
synov a podstromy majú byť symetrické.
